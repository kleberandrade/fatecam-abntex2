\renewcommand{\epigraphname}{EPÍGRAFE}

\begin{epigrafe}[EPÍGRAFE]

\textit{A menos que modifiquemos a nossa maneira de pensar, não seremos capazes de resolver os problemas causados pela forma como nos acostumamos a ver o mundo. Albert Einstein (1879 -- 1955).}

\end{epigrafe}

