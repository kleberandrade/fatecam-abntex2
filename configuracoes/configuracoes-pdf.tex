% ####################################################
% Configurações do PDF Final
% ####################################################
\makeatletter
\hypersetup{%
    portuguese,
    colorlinks=true,   % true: "links" coloridos; false: "links" em caixas de texto
    linkcolor=blue,    % Define cor dos "links" internos
    citecolor=blue,    % Define cor dos "links" para as referências bibliográficas
    filecolor=blue,    % Define cor dos "links" para arquivos
    urlcolor=blue,     % Define a cor dos "hiperlinks"
    breaklinks=true,
    pdftitle={\@title},
    pdfauthor={\@author},
    pdfkeywords={abnt, latex, abntex, abntex2}
}
\makeatother

% ####################################################
% Altera o aspecto da cor azul
% ####################################################
\definecolor{blue}{RGB}{41,5,195}

% ####################################################
% Redefinição dos Labels
% ####################################################
\renewcommand{\algorithmautorefname}{Algoritmo}
\def\equationautorefname~#1\null{Equa\c c\~ao~(#1)\null}

% ####################################################
% Cria um indice remissimo
% ####################################################
\makeindex

% ####################################################
% Hifenização de palavras que nào estão no discionário
% ####################################################
\hyphenation{%
    qua-dros-cha-ve
    Kat-sa-gge-los
}
